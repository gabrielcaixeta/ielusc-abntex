% Arquivo Principal para Monografias dos Cursos de Graduação do IELUSC

% abnTeX2: Modelo de Trabalho Academico em conformidade com 
% ABNT NBR 14724:2011: Informacao e documentacao - Trabalhos academicos -
% Apresentacao

% Adaptado com base no abnTeX2
% Por: Gabriel Caixeta Silva
% E-mail: gabriel.silva@ielusc.br
% ------------------------------------------------------------------------
% ------------------------------------------------------------------------
\documentclass[
	12pt,				% tamanho da fonte
	openright,			% capítulos começam em pág ímpar (insere página vazia caso preciso)
	oneside,			% apenas frente (oneside) ou frente e verso (twoside)
	a4paper,			% tamanho do papel. 
	chapter=TITLE,		% títulos de capítulos convertidos em letras maiúsculas
	section=TITLE,		% títulos de seções convertidos em letras maiúsculas
	%subsection=TITLE,	% títulos de subseções convertidos em letras maiúsculas
	%subsubsection=TITLE,% títulos de subsubseções convertidos em letras maiúsculas
	% -- opções do pacote babel --
	english,			% idioma adicional para hifenização
	spanish,
	french,
	brazil,				% o último idioma é o principal do documento
	]{abntex2}

%Pacotes prinicipais e customização
\usepackage{./Estilo/ielusc}

%Pacote para determinar o número da última página na Ficha Catalográfica
\usepackage{lastpage}

% ---
% Dados da Capa
% ---

\titulo{TÍTULO DO TRABALHO: SUBTÍTULO}
\autor{NOME DO AUTOR}
\local{Joinville}
\instituicao{Associação Educacional Luterana BOM JESUS/IELUSC}
\campus{Centro}
\curso{Sistemas para Internet}
\data{2020}
\fulldata{xx de xxxx de 2020}

% ---
% Folha de Rosto
% ---
\inforosto{Trabalho de conclusão de curso para obtenção do título de graduação em Sistemas para Internet - Tecnólogo apresentado à \imprimirinstituicao}
\orientador{Nome do Orientador}
\orientadorRotulo{Prof. Dr. }
\coorientador{Nome do Co-Orientador}
\coorientadorRotulo{Prof. Dr. }

% ----
% Início do documento
% ----
\begin{document}
% ----
% Elementos Pré-Textuais
% ----
\include{Partes/pretextual}

% ---
% Lista de Abreviaturas e Siglas
% ---
\begin{siglas}
  \acro{OV}{Organização Virtual}
  \acrodefplural{OV}{Organizações Virtuais}
\end{siglas}

\begin{simbolos}
	\SingleSpacing
	\item[\%] Porcentagem
	\item[$D_{ab}$] Distância Euclidiana
	\item[$O(n)$] Ordem de um algoritmo
\end{simbolos}

% ---
% inserir o sumario
% ---

\pdfbookmark[0]{\contentsname}{toc}
\tableofcontents*
\cleardoublepage
% ---

\textual

%Retira o nome do capítulo do header
\pagestyle{eudesc}
\aliaspagestyle{chapter}{eudesc}

% ---

\include{Partes/cap1}

\chapter{Lorem ipsum dolor sit amet}

\lipsum[1]

% ---
\section{Aliquam vestibulum fringilla lorem}
% ---

% ---
\subsection{Aliquam vestibulum fringilla lorem}
\subsubsection{Aliquam vestibulum fringilla lorem}

% ---

\lipsum[1]

\lipsum[2-3]

% ---
% Finaliza o bookmark do PDF
% ---
\bookmarksetup{startatroot}% 
% ---

% ----------------------------------------------------------
% ELEMENTOS PÓS-TEXTUAIS
% ----------------------------------------------------------
\postextual

% ----------------------------------------------------------
% Referências bibliográficas
% ----------------------------------------------------------
\bibliography{references}

% ----------------------------------------------------------
% Glossário
% ----------------------------------------------------------
%
% Consulte o manual da classe abntex2 para orientações sobre o glossário.
%
%\glossary

% ----------------------------------------------------------
% Apêndices
% ----------------------------------------------------------
\begin{apendicesenv}
	\include{Partes/apeA}
\end{apendicesenv}

% ----------------------------------------------------------
% Anexos
% ----------------------------------------------------------
\begin{anexosenv}
	\include{Partes/aneA}
\end{anexosenv}

\end{document}
