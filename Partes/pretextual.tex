%!TEX root = ../Principal.tex
%Capa do Trabalho
\imprimircapa

%Folha de Rosto
%* indica que tem ficha catalográfica
\imprimirfolhaderosto
% ---
% Caso a Biblioteca do IELUSC forneça, utilize o comando
% ---
% \begin{fichacatalografica}
%     \includepdf{fig_ficha_catalografica.pdf}
% \end{fichacatalografica}
% ---
% Geração da Ficha Catalográfica Via LaTeX
% ---
% \begin{fichacatalografica}
%  	\vspace*{\fill}					% Posição vertical
% % 	\begin{center}					% Minipage Centralizado
% % 	\begin{minipage}[c]{12.5cm}		% Largura
% % 	\imprimirautor

% % 	\hspace{0.5cm} \imprimirtitulo  / \imprimirautor. --
% % 	\imprimirlocal, \imprimirdata-
	
% % 	\hspace{0.5cm} \pageref{LastPage} p. : il. (algumas color.) ; 30 cm.\\
	
% % 	\hspace{0.5cm} \imprimirorientadorRotulo~\imprimirorientador\\
	
% % 	\hspace{0.5cm}
% % 	\parbox[t]{\textwidth}{\imprimirtipotrabalho~--~\imprimirinstituicao,
% % 	\imprimirdata.}\\
	
% % 	\hspace{0.5cm}
% % 		1. Tópico 01.
% % 		2. Tópico 02.
% % 		I. Prof. Dr. xxxxx.
% % 		II. Associação Educacional Luterana BOM JESUS/IELUSC.
% % 		III. Centro.
% % 		IV. identificação xxxx\\ 			
	
% % 	\hspace{8.75cm} CDU 02:121:005.7\\
	
% % 	\end{minipage}
% % 	\end{center}
 
% \end{fichacatalografica}

% ---
% Folha de Aprovação
% ---
% Exemplo de folha de aprovação antes da Banca. Após isso, incluia o pdf digitalizado com as assinaturas%
% \includepdf{folhadeaprovacao_final.pdf}
\begin{folhadeaprovacao}
	\begin{center}
	    \vspace*{2cm}
		{\ABNTEXchapterfont\bfseries\imprimirautor}
		\vspace{6em}

			\ABNTEXchapterfont\bfseries\imprimirtitulo
		
	\end{center}
	\hspace{.45\textwidth}
    \begin{minipage}{.5\textwidth}
    
    		\vspace{2em}
    		Trabalho de conclusão de curso para obtenção do título de graduação em Sistemas para Internet - Tecnólogo apresentado à \imprimirinstituicao
    	
	\end{minipage}
	
 	\vspace{3em}
	\noindent
	{\bfseries Banca Examinadora:}
	\assinatura{\textbf{\imprimirorientador} \\ \imprimirinstituicao} 
	\assinatura{\textbf{Prof. Nome do Professor} \\ \imprimirinstituicao}
    \assinatura{\textbf{Prof. Nome do Professor} \\ \imprimirinstituicao}

    \vspace*{\fill}
    \begin{center}
    	\imprimirlocal,\,\imprimirfulldata
    \end{center}
    \vspace*{1cm}
\end{folhadeaprovacao}

% ---
% Dedicatória
% ---
% \begin{dedicatoria}				
% Dedico este trabalho aos meus familiares, amigos, colegas e professores que me acompanharam e me deram forças nessa magnífica trajetória.  
% \end{dedicatoria}

% ---
% Agradecimentos
% ---
% \begin{agradecimentos}
% Gostaria de agradecer...

% Aqui devem ser colocadas os agradecimentos às pessoas que de alguma forma contribuíram para a realização do trabalho.
% \end{agradecimentos}

% ---
% Epígrafe
% ---
% \begin{epigrafe}	
% ``Independentemente das circunstâncias, devemos ser sempre humildes, recatados e despidos de orgulho.''
% \\
% \par
% Dalai Lama 
% \end{epigrafe}

% ---
% RESUMOS
% ---

% ---
% Ao usar o modo twoside (anverso e verso) o resumo não se posiciona na página ímpar.
% Dessa forma, deve-se forçar o resumo para iniciar em uma página ímpar, usando o seguinte comando:
% ---

%\newpage\null\thispagestyle{empty}\newpage

% Português
\begin{resumo}
 O resumo deve ressaltar o
 objetivo, o método, os resultados e as conclusões do documento. A ordem e a extensão
 destes itens dependem do tipo de resumo (informativo ou indicativo) e do
 tratamento que cada item recebe no documento original. O resumo deve ser
 precedido da referência do documento, com exceção do resumo inserido no
 próprio documento. (\ldots) As palavras-chave devem figurar logo abaixo do
 resumo, antecedidas da expressão Palavras-chave:, separadas entre si por
 ponto e finalizadas também por ponto.

 \vspace{\onelineskip}
    
 \noindent
 \textbf{Palavras-chaves}: latex, abntex e editoração de texto.
\end{resumo}

% ---
% Ao usar o modo twoside (anverso e verso) o resumo não se posiciona na página ímpar.
% Dessa forma, deve-se forçar o resumo para iniciar em uma página ímpar, usando o seguinte comando:
% ---

%\newpage\null\thispagestyle{empty}\newpage

% Inglês
\begin{resumo}[Abstract]
 \begin{otherlanguage*}{english}
	Resumo em inglês
   \vspace{\onelineskip}
 
   \noindent 
   \textbf{Key-words}: latex, abntex e text editoration.
 \end{otherlanguage*}
\end{resumo}

% ---
% Lista de Figuras
% ---
\pdfbookmark[0]{\listfigurename}{lof}
\listoffigures*
\cleardoublepage
% ---

% ---
% Lista de Tabelas
% ---
\pdfbookmark[0]{\listtablename}{lot}
\listoftables*
\cleardoublepage
